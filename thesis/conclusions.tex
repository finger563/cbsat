\chapter{Conclusions and Future Work}
\label{ch:conclusions}

We have described in this thesis the aspects of Cyber-Physical
Systems(CPS) analysis, design, development, and integration we are
addressing.  We have provided descriptions of the relevant related
work in this area, covering both the design-time modeling, analysis,
and performance prediction for networked, distributed, CPS
applications and the run-time monitoring and management of application
and CPS network resources.

Subsequently, the completed research towards precise network
performance prediction, \shorttool/, was presented. 

First, the formalization for the modeling and analysis semantics and
techniques were defined, building off of the $(\wedge,+)-calculus$
used by Network Calculus.  Models of systems and applications were
presented and convolution ($\otimes$) of application profiles with
system profiles was defined.  Using $(\wedge,+)-calclulus$, the
computation of delay and backlog bounds were defined.

Given the definition of the fundamental operations of \shorttool/,
analysis of periodic systems was presented.  We described how periodic
data rate profiles can be time-integrated to produce repeating data
profiles as functions of time.  We proved that the minimum amount of
time for which the system and its applications must be analyzed to
determine if there is unbounded buffer growth is two hyperperiods.

Using experimental system data, we determined the benefit of
\shorttool/ versus similar techniques such as Network Calculus.  We
showed how our techniques provide more accurate predictions with
respect to the actual system but are still conservative predictions.  

We then showed how a model of MAC protocols, such as TDMA, could be
incorporated into the system and application models.  Using these
modles, we analytically derived equations for the effects of such
protocols on the predicted delay and backlog bounds.

The mathematical operations of \shorttool/ were extended to support
compositional system analysis by defining the concepts of profile
addition and subtraction.  For this compositional analysis, the
concept of profile priority was introduced to determine service
precedence by the transmitting node between two profiles.

Since latency is such a critical aspect of networking systems, we
introduced semantics for modeling the delay of network links as a
linear, continuous function of time.  Convolution of a profile with a
delay profile was introduced and its effects on the profile's
periodicity were analyzed.

To support more complex systems which include nodes that can act as
routers and forwarders for traffic from other nodes, we presented an
algorithm that uses the concepts we developed for delay analysis and
compositional analysis to iteratively analyze a system which contains
statically routed traffic.  Experimental validation of this
integrated, system-level analysis was provided to demonstrate the
accuracy and precision of the analysis techniques.  

To support experimental validation and run-time testing, we developed
code generators that generate traffic producer/consumer and
measurement code into the component models we defined.  Using these
producers and consumers, which operate based on the same profiles used
for design-time analysis, we ran experiments which corroborated our
analysis results.

Finally, we extended our traffic producer/consumer code to enable
management of the network traffic by the communications middleware.
Detection code was developed for the receivers to detect when and
which senders were overflowing the receiver's buffer and use an
out-of-band communications channel to inform the sender's middleware
to limit the sender's data production.  

\subsection{Future Work}
\label{subsec:future_work}
In this work we have described the beginning of precise, comprehensive
network system performance analysis and prediction.  However, we could
not possibly cover the modeling and analysis of all possible system
configurations, communication protocols, or interaction paradigms.
Furthermore, we have examined the affect certain system configuration
parameters or modeling choices have on our analysis techniques and
results, but such examination is not exhaustive.

Extending this work would focus on these areas in the following ways:

\begin{itemize}
  \item Modeling and analysis support for more (commonly used)
    transmission protocols
  \item Relaxing modeling constraints and analyzing the affect of such
    relaxation on the predicted performance of the system
  \item Investigating run-time implementation alternatives and data
    analysis techniques
\end{itemize}

Towards the first area, a primary extension to the modeling framework
would be the support of retun-path communications and the impact they
have on the system.  Such modeling and analysis would allow for the
inclusion of TCP-based communications and could pave the way for
reactive system analysis using these techniques.  

Similarly, one of the model constraints which can be relaxed and
analyzed is the system-wide time-synchronization constraint.  If that
constraint is relaxed such that the nodes are known be re-synchronized
periodically with some predictable drift, then such behavior can be
directly analyzed similarly to the TDMA analysis.  From this
information, maximum deviations on the required buffer and delay can
be calculated, similarly to the deviations calculated for TDMA
systems.


