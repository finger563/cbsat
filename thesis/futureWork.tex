\chapter{Future Work}
\label{ch:future_work}

In this work we have described the beginning of precise, comprehensive
network system performance analysis and prediction.  However, we could
not possibly cover the modeling and analysis of all possible system
configurations, communication protocols, or interaction paradigms.
Furthermore, we have examined the affect certain system configuration
parameters or modeling choices have on our analysis techniques and
results, but such examination is not exhaustive.

Extending this work would focus on these areas in the following ways:

\begin{itemize}
  \item Moding and analysis support for more (commonly used)
    transmission protocols
  \item Relaxing modeling constraints and analyzing the affect of such
    relaxation on the predicted performance of the system
  \item Investigating run-time implementation alternatives and data
    analysis techniques
\end{itemize}

Towards the first area, a primary extension to the modeling framework
would be the support of retun-path communications and the impact they
have on the system.  Such modeling and analysis would allow for the
inclusion of TCP-based communications and could pave the way for
reactive system analysis using these techniques.  

Similarly, one of the model constraints which can be relaxed and
analyzed is the system-wide time-synchronization constraint.  If that
constraint is relaxed such that the nodes are known be re-synchronized
periodically with some predictable drift, then such behavior can be
directly analyzed similarly to the TDMA analysis.  From this
information, maximum deviations on the required buffer and delay can
be calculated, similarly to the deviations calculated for TDMA
systems.
